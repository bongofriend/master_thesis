\section{Zukünftige Aussichten}

Die fortschrittlichen Entwicklungen im Bereich des maschinellen Lernens (ML) und speziell der Large Language Models (LLMs) eröffnen neue Perspektiven für die automatisierte Erkennung von Design Patterns in Quellcode. In diesem Feld wurden zum Zeitpunkt des Verfassens der Arbeit Fortschritte getätigt, da LLMs das Potenzial geben, die Effizienz und Genauigkeit bei der Identifizierung von Design Patterns erheblich zu verbessern. Im Folgenden werden die zukünftigen Aussichten dieser Technologie beleuchtet.

\begin{enumerate}
    \item \textbf{Erweiterte Erkennungskapazitäten}: Mit der zunehmenden Verfeinerung von LLMs ist zu erwarten, dass ihre Fähigkeit, komplexe Muster und Abstraktionen im Quellcode zu erkennen, deutlich zunimmt. Diese Modelle können aus einer umfangreichen Datenmenge lernen und somit eine breite Palette von Design Patterns identifizieren, die in verschiedenen Programmiersprachen und -stilen zum Einsatz kommen. Die Flexibilität von LLMs ermöglicht es ihnen, auch seltene oder weniger dokumentierte Patterns zu erkennen, die herkömmliche Methoden möglicherweise übersehen.
    \item \textbf{Verbesserung der Präzision und Reduzierung von Fehlalarmen}: Durch das Training mit großen Datensätzen können LLMs nicht nur eine Vielzahl von Patterns erkennen, sondern auch den Kontext, in dem diese Patterns verwendet werden, besser verstehen. Dies führt zu einer höheren Präzision bei der Erkennung und einer signifikanten Reduzierung von Fehlalarmen. Die Fähigkeit, den Kontext zu berücksichtigen, ist besonders wichtig, da viele Design Patterns nur in bestimmten Situationen angemessen sind. LLMs können feine Unterschiede im Code erfassen, die darauf hinweisen, ob ein bestimmtes Pattern tatsächlich beabsichtigt ist oder nicht.
    \item \textbf{Automatisierte Verbesserungsvorschläge und Refactoring}: Zukünftige Entwicklungen könnten LLMs befähigen, nicht nur existierende Patterns zu erkennen, sondern auch Verbesserungsvorschläge zu machen. Basierend auf der erkannten Implementierung eines Design Patterns könnten diese Modelle Empfehlungen für ein effizienteres oder klareres Pattern geben, das in den aktuellen Code eingeführt werden kann. Ferner ist das Potenzial für automatisiertes Refactoring beträchtlich, wobei LLMs Vorschläge für Code-Umstrukturierungen machen können, um Design Principles wie SOLID besser einzuhalten.
    \item \textbf{Interaktive Entwicklungsumgebungen}: Die Integration von LLMs in Entwicklungsumgebungen und IDEs (Integrated Development Environments) könnte zu einer interaktiveren und unterstützenden Codierungserfahrung führen. Entwickler könnten in Echtzeit Feedback zu den von ihnen verwendeten Design Patterns erhalten, einschließlich Hinweisen zur Anwendung und möglichen Optimierungen. Diese Art der direkten Integration fördert ein tieferes Verständnis für Design Patterns und unterstützt Entwickler dabei, best practices effektiver in ihren Code zu integrieren.
\end{enumerate}