\section{Fazit}

Zusammenfassend ist das Ergebnis dieser Arbeit als negativ zu beurteilen. Zwar ist der Algorithmus für das Zuweisen der Entwurfsmuster in der Lage, teilweise das korrekte Design Pattern zuzuordnen, jedoch ist dessen Aussagekraft von der Leistung des Klassifizierers abhängig. Wie in der Sektion der Evaluation erläutert, ist diese als mangelhaft zu beurteilen.
Die Problematik besteht darin, dass für das Trainieren von Machine Learning Modellen für Klassifikationsaufgaben keine definitiven Prozesse existieren.
Dies gilt vor allem für die Bestimmung der Features, die durch den verfügbaren Datensatz und dessen Qualität und Umfang limitiert sind. 
Wie in Sektion~\ref{dataset_analysis} zu sehen ist, ist das Verhältnis zwischen Datenpunkten und Design Pattern von dem Umfang der einzelnen Implementierungen abhängig. Dadurch, dass die Design Patterns so unbalanciert im Datensatz repräsentiert sind, kann es vorkommen, dass einige Design-Pattern-Rollen überrepräsentiert, während andere unterrepräsentiert sind. Zudem ist die Konfiguration der Hyperparameter ebenfalls eine weitere Methode, um die Klassifikationsleistung zu verbessern. Zwar stellen automatisiertes Hyperparameter-Tuning und der Einsatz von Kreuzvalidierung eine Option dar, um die Leistung des Klassifizierers zu optimieren und zu validieren, jedoch ist das beste Ergebnis dadurch nicht garantiert. Es besteht immer Wahrscheinlichkeit, dass die bestmögliche Konfiguration an Hyperparameter-Werten außerhalb des Suchraumes liegt.

Insgesamt gesehen ist es davon auszugehen, dass verschiedene Aspekte und Optionen existieren, um die hier vorgeschlagene Methode weiter zu verbessern. 
Entweder kann unter anderem ein anderer Katalog an Metriken als Features genutzt werden oder es werden andere Klassifizierer verwendet.