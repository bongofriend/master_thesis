\section{Herausforderungen und Probleme bei der Erkennung von Design Patterns}

\subsection{Variabilität der Implementierung}

Design Patterns stellen in der Software-Entwicklung bewährte Lösungsmuster für bereits begegnete Herausforderung dar. Aufgrund der abstrakten und wiederverwendbaren Natur der Entwurfsmuster,
muss für diese eine konkrete Implementierung definiert werden, die von dem Einsatzfall, Kontext und anderen Faktoren wie verwendeter Programmiersprache, Bibliotheken und Erfahrungsstand des Software-Entwicklers.
Dadruch, dass jedes Entwurfsmuster einen konzeptionellen Rahmen darstellt und jede Implementierung von nicht statischer Außenfaktoren beeinflusst wird, resultiert dies in einem breiten Spektrum an Implementierungen für ein gegebenes Entwurfsmuster.
Aus diesem Grund ist eine Definition einer starren Definition eines Design Patterns, was als Startpunkt und Referenz für die Erkennung des jeweiligen Entwurfsmusters dienen könnte, nicht möglich. 
Deshalb ist eine definitive Antwort auf die Frage, ob eine betrachte Mikroarchitektur eine Instanz eines Entwurfsmusters, nicht beantwortbar, weshalb die Antwort von automatisierten Prozessen von Design Patterns eher mit einem Wert besteht,
welches die Ähnlichkeit zu einem Design Pattern beschreibt.

\subsection{Steigende Komplexität und Skalierbarkeit des Software-Systems}

\subsection{Iterative Evolution des Quellcodes}

\subsection{Mangel an expliziter Dokumentation}

