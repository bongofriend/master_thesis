\section{Herausforderungen und Probleme bei der Erkennung von Design Patterns}

In diesem Abschnitt der Arbeit werden mögliche Herausforderungen diskutiert, die bei dem Entwerfen des Prozesses für die Erkennung von Entwurfsmustern auftreten können. 

\subsection{Variabilität der Implementierung}

Design Patterns stellen in der Software-Entwicklung bewährte Lösungsmuster für bereits begegnete Herausforderung dar. Aufgrund der abstrakten und wiederverwendbaren Natur der Entwurfsmuster,
muss für diese eine konkrete Implementierung definiert werden, die von dem Einsatzfall, Kontext und anderen Faktoren wie verwendeter Programmiersprache, Bibliotheken und Erfahrungsstand des Software-Entwicklers.
Dadruch, dass jedes Entwurfsmuster einen konzeptionellen Rahmen darstellt und jede Implementierung von nicht statischer Außenfaktoren beeinflusst wird, resultiert dies in einem breiten Spektrum an Implementierungen für ein gegebenes Entwurfsmuster.
Aus diesem Grund ist eine Definition einer starren Definition eines Design Patterns, was als Startpunkt und Referenz für die Erkennung des jeweiligen Entwurfsmusters dienen könnte, nicht möglich. 
Deshalb ist eine definitive Antwort auf die Frage, ob eine betrachte Mikroarchitektur eine Instanz eines Entwurfsmusters, nicht beantwortbar, weshalb die Antwort von automatisierten Prozessen von Design Patterns eher mit einem Wert besteht,
welches die Ähnlichkeit zu einem Design Pattern beschreibt. Um einen zufriedenstellenden Wert für diese Frage zu liefern, bedarf es eines bereiten Spektrums an Implementierungsvariationen als Referenz für die Erkennung.

\subsection{Steigende Komplexität von Software-Systemen}

Die steigende Komplexität von Software-Systemen stellt eine erhebliche Herausforderung bei der Erkennung von Entwurfsmustern in Entwurfsmustern dar.
Dies ist besonders bei langjährigen Software-Projekten der Fall, an denen über die Zeit konstante Änderungen wegen Wartung und neuen bzw. geänderten Anforderungen unterliegen.
Diese Art von Software-Projekten tendieren dazu, dass mit der Zeit deren Komplexität zunimmt~\cite[S. 7]{Suh-2010}. 
Bei kleineren Software-Projekten mit geringem Umfang und Komplexität sind Entwurfsmuster leichter zu erkennen und zu implementieren.
Dahingegen bei langjährigen Software-Projekten steigt mit wachsender Gesamtkomplexität die Komplexität der angewendeten Entwurfsmuster in deren Quellcode, wodurch die Identifizierung dieser proportional mitsteigt.
Entwurfsmuster werden durch diese Entwicklung weiter modifiziert und angepasst, womit diese von der ursprünglichen leichter zu identifizierbaren Iterationen weiter abweichen. 
Deshalb beinhaltet die Erkennung von Entwurfsmustern nicht nur auf die momentane Iteration, sondern auch die Erfassung derer historischen Evolution und die Entwicklung dieser innerhalb der Codebasis.

\newpage

\subsection{Iterative Evolution des Quellcodes}

Damit ein Software-System seine Anforderungen im Verlauf dessen Lebenszyklus in einer zufriedenstellenden Art und Weise erfüllen kann, muss dieses adaptieren, um diesen Anforderungen gerecht zu werden~\cite[S. 108]{10.1007/BFb0017737}.
Dies hat zu Folge, dass dessen Quellcode iterativen Änderungen unterliegt. Ursprüngliche Implementierungen in der Codebasis werden analysiert und es wird überprüft, ob diese ihre Aufgaben zufriedenstellend erfüllen oder nicht.
Falls nicht, werden diese so modifiziert, sodass diese Anforderungen auf erwartete Art und Weise erfüllt werden können. Implementierungen von angewandten Design Patterns als Teil des Quellcodes unterliegen ebenfalls dieser Analyse.
Diese werden als Teil des Analyseprozesses genauer betrachtet und werden nach Bedarf modifiziert und angepasst. Diese Entwicklung führt wie das im vorherigen Abschnitt diskutierten Fall, dass Entwurfsmuster von ihrer ursprünglichen leichter zu identifizierbaren
Iteration weiter abweichen und die Erkennung von Entwurfsmustern nicht nur die momentane Implementierung, sondern auch die historische Entwicklung berücksichtigen werden muss. In automatisierten Prozess der Erkennung von Entwurfsmustern kann dieser Aspekt nur bedingt berücksichtigt,
weil das Einschließen der Historie der betrachtenden Implementierung und dessen Kontext in der Codebasis nicht pauschal und in einer allgemeinen Ansicht betrachtet werden kann. 

\subsection{Mangel an expliziter Dokumentation}

Das Erstellen und Warten von Dokumentation für Software-Systeme ist eine Tätigkeit, die von Software-Entwicklern als wichtig eingestuft wird, jedoch in diese Tätigkeit relativ wenig Zeit investiert wird~\cite[S, 162]{zhi2015cost}.
Dies ist die Folge der Dominanz des agilen Software-Entwicklungsprozesses, in dieser die Verwendung von Zeit und Ressourcen für die Dokumentation eher als Verschwendung betrachtet wird, da diese keinen direkten Mehrwert für die Auslieferung des Software-Produktes an den Endkunden liefert~\cite[S. 159]{zhi2015cost}.
Das dies das Software-System komplett betrifft, sind Design Patterns in dessen Codebasis ebenfalls betroffen. Diese werden meist nicht direkt gekennzeichnet. 
Zwar kann durch Nomenklatur und Kontrollfluss indirekt Rückschlüsse auf die potenziellen Entwurfsmuster abgeleitet werden, jedoch erfordert dies konkretes Fachwissen und Erfahrungen, die nicht von jedem Software-Entwickler erfüllt werden kann.
Bei automatisierten Prozessen für die Erkennung von Entwurfsmustern kann diese berücksichtigt werden, sollte aber nicht als alleiniger Faktor bei dem Identifikationsprozess dienen.\\

Aufgrund der Variation an Implementierungsmöglichkeiten, Änderungen im Quellcode und Mangel an Dokumentation ist die manuelle Identifikation von Entwurfsmustern in Quellcode ein Prozess dar,
in der einen gewissen Grad an Mitdenken erfordert. Im nächsten Abschnitt der Arbeit werden bereits entwickelte Verfahren betrachtet, die das Mitdenken bis zu einem gewissen Grad automatisieren und dieses als Teil des Prozesses mitinludieren.




