\section{Extrahierte Features}

Im Kontext dieser Arbeit werden Design Patterns als eine Summe von Substrukturen definiert, zu diesem jeweils eine Rolle zugeordnet wird.
Im Sinne dieser ist eine Menge an Erkennungsmerkmalen oder Features notwendig, welche Rückschlüsse auf die jeweilige Rolle ermöglicht. 
Dabei müssen sich die Aspekte für Struktur, Verhalten und Relation zu anderen Komponenten in ausgewählten Features widerspiegeln.
Die in einem vorherigen Abschnitt erläuterten Machine Learning Verfahren nutzen verschiedene Arten von Features, um diese Aspekte zu encodieren.
Im Kern läuft es darauf hinaus, die Erkennungsmerkmale in eine numerische Repräsentation zu transformieren, sodass ein Klassifizierer mit diesen umgehen kann.
Deswegen werden in dieser Arbeit die Merkmale der jeweiligen Instanzen der Rollen als numerische Vektoren dargestellt, die die Aspekte der Struktur, Verhalten und Relation als Metriken encodieren.

Die Entscheidung, Metriken als Features für Rollen zu verwenden, bietet mehrere Vorteile. Zu einem ermöglicht der Einsatz von Metriken, Eigenschaften der Implementierung der jeweiligen Rollen in einer abstrakten Form darzustellen.
Dadurch kann die Identifikation der Rollen unabhängig von Programmiersprache und Projektstruktur erfolgen. Die Ermittelung der jeweiligen Metriken erfolgt über statische Codeanalyse, die durch Parallelisierung in der Lage ist,
einen großen Umfang an Quelldateien in relativ kurzer Zeit zu bearbeiten. Der wohl größte Verteil, die Metriken erbringen, ist Transparenz und Reproduzierbarkeit.
Bevor ein Wert zu einer Metrik ermittelt werden kann, muss zunächst definiert werden, was die jeweilige Metrik aussagt und wie diese ermittelt wird. Durch eine automatisierte und strikte Befolgung der Definition der Berechnung kann durch statische Codeanalyse in mehreren Durchläufen der gleiche Wert für die Metrik im Kontext der Rolleninstanz bestimmt werden.
Dadurch entsteht ebenfalls eine Transparenz, wie die Werte ermittelt werden. Bei den hier erläutern Verfahren, die die Features der Entitäten aus dem Quellcode grafisch oder in natürlicher Sprache encodiert haben, müssen diese zusätzlich in eine numerische Form gebracht werden.
Dies erfolgt meist durch das Inkludieren eines zusätzlichen Modells. Zu einem erfordert dies zusätzliches Training und Validierung des encodierenden Modells, welches je nach Modell und Umfang des Datensatzes auch noch Zeit und Rechenleistung in Anspruch nimmt. Dazu ist die Leistung des Klassifizierers abhängig von der Qualität der Resultate des encodierenden Modells. Mögliche Leistungsdefizite des Encodierens propagieren zu der Klassifizierungskapazität des gesamten Verfahrens.
Zwar ist die Ermittlung von Metriken davon nicht ausgeschlossen, jedoch sind Rückschlüsse möglich, wie diese Werte zustande gekommen sind. Dahingegen sind die Modelle, die die Features transformieren, als Black Box zu betrachten. Die Werte innerhalb des Modells, die die Transformation der Eingabedaten beeinflussen, sind für das Menschenauge größtenteils nur schwer verstehbar und bieten keinen Rückschluss, wie diese berechnet worden sind.
Zusammenfassend ermöglichen Metriken als Feature eine Unabhängigkeit von Programmiersprache und Projektstruktur, Skalierbarkeit, Reproduzierbarkeit und Transparenz, die bei dem weiteren Verfeinern der hier vorgestellten Methodik wünschenswert sind.

\pagebreak

Um die Kennmerkmale hier zu ermitteln, erfordert dies ein Katalog an Metriken, die angegeben, wie diese ermittelt werden und was diese aussagen. Hierbei werden Metriken verwendet, die die Aspekte der Struktur, des Verhaltens und der Relation widerspiegeln.
Dazu werden Metriken verwendet, die bereits in anderen Arbeiten verwendet wurden, die die Struktur repräsentieren und Metriken von Chidamber et al., die Einblicke in Verhalten und Relation bieten~\cite{chidamber1994metrics}:


\begin{table}[H]
    \begin{tabular}{|c|p{0,45\linewidth}|p{0,35\linewidth}|c|}
        \hline
        Kürzel & Langform &Beschreibung & Datentyp\\
        \hline
        COA & COUNT\_OF\_ABSTRACT\_METHODS & Anzahl der abstrakten Methoden in einer Klasse & int\\
        COF & COUNT\_OF\_FIELDS & Anzahl der Felder in einer Klasse & int\\
        COM & COUNT\_OF\_METHODS & Anzahl der Methoden in einer Klasse & int\\
        COF & COUNT\_OF\_OBJECT\_FIELDS & Anzahl der Felder in einer Klasse mit Objekttyp & int\\
        COPC & COUNT\_OF\_PRIVATE\_CONSTRUCTORS & Anzahl der privaten Konstruktoren & int\\
        COPF & COUNT\_OF\_PRIVATE\_FIELDS & Anzahl der privaten Felder & int\\
        COSF & COUNT\_OF\_STATIC\_FIELDS & Anzahl an statischen Feldern & int\\
        CBO & COUPLING\_BETWEEN\_OBJECTS & Anzahl der Abhängigkeiten in der Klasse nach Chidamer et al. & int\\
        DIT & DEPTH\_OF\_INHERITANCE & Anzahl der Elterklassen einer Klasse nach Chidamer et al. & int\\
        RPC & RESPONSE\_FOR\_A\_CLASS & Anzahl der Methoden nach Chidamer et al, die direkt durch Klassenmethoden aufgerufen werden & int\\
        WMC & WEIGHTED\_METHOD\_CLASS & Zyklomatische Komplexität nach McCabe~\cite{mccabe1976complexity} & float\\
        \hline
    \end{tabular}
    \caption{ausgewählte Metriken als Features}
    \label{tab:features}
\end{table}

Der in Tabelle~\ref{tab:features} definierte Katalog an Metriken dient für die Beantwortung der Untersuchungsfrage~\ref{RQ4} und die Identifikation an Rollen im Kontext des jeweiligen Design Patterns der Untersuchungsfrage~\ref{RQ3}.



