\section{Verfügbare Datensätze}\label{dataset}
Um ein Maschine-Modell zu trainieren, benötigt es einen Datensatz mit passenden gekennzeichneten Klassen.
In der Erkennung von Entwurfsmustern existiert zu dem Zeitpunkt der Verfassung dieser Arbeit keine standardmäßige oder weit akzeptierter Datensatz.
Jedoch existieren Versuche, solch einen Datensatz zu etablieren. Im Folgendem wird eine Auswahl an möglichen existierenden Datensätzen aufgelistet~\cite[S. 104 - 105]{phdthesis}:

\begin{description}
    \item Pattern-like Micro-Architecture Repository (P-MArt): Dieser Datensatz ist der einzige der hier aufgelisteten, welcher Peer-Reviews unterzogen wurde. Zwar bestehen die Autoren nicht darauf, dass alle möglichen Instanzen von Entwurfsmustern in den Software-Systemen identifiziert wurde, haben diese Zuversicht, dass die Mehrheit der Instanzen durch mehrfache Analyse durch mehrere Teams erkannt wurde. 
    \item DEsign pattern Evaluation BEnchmark Environment (DEEBEE): Dieser Datensatz beinhaltet Resulate aus automatischer und einer manuellen Analyse von fünf Software-Systemen auf Entwurfsmustern.
    \item Design Pattern Benchmark (DPD): DPD beinhaltet die Ergebnisse von Software-Systemen durch mehrere Software-Werkzeuge. Zudem sind die Daten des Datensatzes durch eine Webseite verfügbar, in der nach Instanzen von Entwurfsmustern in Software-Systemen gesucht werden kann. Zusätzlich können Ergebnisse evaluiert und durch öffentlich zugängliche Abstimmung einer Instanz das passende Design Pattern zugeordnet werden.
    \item PattErn Repository and Components Extracted from OpeN Source software (PERCERONS): PERCERONS ist das derzeit von dem Umfang her der größte verfügbare Datensatz mit über 537 analysierten Software-Systemen. Die Ergebnisse wurden durch das Kombinieren der Resultate zweier Software-Ergebnisse ermittelt.
    \item Software Engineering Research Center benchmark (SERC): SERC wurde konstruiert, um als Vergleichsmaßstab für ein Software-Werkzeug zu dienen, welches von den gleichen Autoren stammte. Dabei werden die Ergebnisse von mehreren anderen Werkzeugen mit den Ergebnissen ihres eigenes kombiniert.
\end{description}

Die Problematik bei der Ermittelung eines geeigneten Datensatzes besteht darin, dass der Umfang der analysierten Software-Systeme sich in Grenzen hält und meist eher als Maßstab für die Leistung der selbst entwickelten Methode verwendet wird, wodurch Bias zu der eignen Methode nicht ausgeschlossen werden kann.
Zudem sind die hier aufgeführten Datensätze bis auf einige Ausnahmen durch die Kombination von mehreren Software-Werkzeugen entstanden ist. Durch solch ein Verfahren ist zwar die Analyse von mehreren Software-Systemen in einer relativ kurzen Zeitspanne möglich, jedoch wäre eine manuelle Verifizierung der Ergebnisse aufgrund von hohen Zeit- und Leistungsaufwands schwer umsetzbar.
Dahingegen wäre das ideale Vorgehen, das solch ein Datensatz durch manuelles Identifizieren von Entwurfsmustern in Quellcode durch die Analyse von mehreren Software-Systemen durch ein Experten-Team erstellt wird. Jedoch ist solch ein Verfahren zeitaufwändig und erfordert ein tiefes Verständnis des jeweiligen Software-Systems, welches durch mangelnde oder obsolete Dokumentation eine Hürde darstellt.
Die nächst bestmögliche Alternative dazu ist die Option des Crowdsourcings wie in DPD. Das Problem ist hier, dass die Qualifikation der Leute, die Instanzen von Entwurfsmustern identifizieren und über die korrekte Zuweisung abstimmen, nicht garantiert. Zudem erfordert dies Publikum mit der Bereitschaft dies zu tun, welches auch nicht garantiert werden kann.
Bei dem im Kontext dieser Arbeit vorgestellten Verfahren für die Ermittelung von Entwurfsmustern wird meist ein Datensatz im kleinen Umfang erstellt. Auf Anfrage auf Zugang zu einigen Datensätzen gibt zu dem Zeitpunkt zur Verfassung dieser Arbeit keine Antwort seitens der Autoren. Zudem sind die Webseiten, worüber auf die Datensätze zugegriffen werden kann, nicht mehr verfügbar und der Zugang über ein Archivierungsdienstleistung wie der Wayback Maschine war ebenfalls nicht möglich. 
Aus den hier aufgelisteten Gründen wird im Zuge dieser Arbeit der P-MArt Datensatz verwendet und somit die Untersuchungsfrage~\ref{RQ2} zu beantworten. Dieser ist zum Zeitpunkt des Verfassens dieser Arbeit zugänglich und ist der einzige, der Peer-Reviews unterzogen wurde. In der Sektion der Implementierung wird dieser genauer betrachtet.
