  \documentclass[twoside=false, %  doppelseitiger Druck
    DIV=15,% DIV Faktor für Satzspiegelberechnung, sie Doku zu KOMA Script
    BCOR=15mm, % Bindekorrektur
    chapterprefix=false,
    headinclude=true,
    footinclude=false,
    pagesize,%         write pagesize to DVI or PDF
    fontsize=11pt,%             use this font size
    paper=a4,%          use ISO A4
    bibliography=totoc,%         write bibliography-chapter to table of contents
    index=totoc,%         write index-chapter to table of contents
    cleardoublepage=plain,% \cleardoublepage generates pages with pagestyle empty
     headings=big,%       A4/B5
    listof=flat,%        improved list of tables
    numbers=noenddot
  ]{scrbook}

\usepackage[utf8]{inputenc}
\usepackage{makeidx}
\usepackage{amsfonts}
\usepackage[slantedGreek,sc]{mathpazo}  % Schriftart Palatino
% \usepackage{lmodern}    % statt mathpazo, falls CM Fonts verwendet werden sollen
%\usepackage{mathptmx}    % statt mathpazo, falls Times  verwendet werden soll
\usepackage[scaled=.95]{helvet}
\usepackage{courier}
\usepackage[T1]{fontenc}
\usepackage{textcomp}
\usepackage{amsmath}            % standard math notation (vectors/sets/...)
\usepackage{bm}        % standard math notation (fonts)
\usepackage{fixmath}        % standard math notation (fonts)
\usepackage{graphicx}
\usepackage[facing=yes]{floatrow}       % mehrere Gleitobjekte nebeneinander/caption neben Bild/Tabelle
\usepackage[labelfont=bf,sf,font=small,labelsep=space,format=plain]{caption}
\usepackage{subcaption}
\usepackage{scrlayer-scrpage}
% \usepackage{pstool}  % einbinden falls psfrag verwendet werden soll
\usepackage{epstopdf}
\usepackage[ngerman]{babel}
\usepackage{ellipsis}  % Korrigiert den Weißraum um Auslassungspunkte
\usepackage{microtype}  % optischer Randausgleich etc.

\usepackage{xcolor}         % z.B. für schattierte Boxen
\usepackage{framed}			% shaded Umgebung
\definecolor{shadecolor}{gray}{.85}%

% Links im PDF
\usepackage[colorlinks=false,
            pdfborder={0 0 0},
            breaklinks=true]
            {hyperref}


%\typearea[current]{calc}


% Einstellungen für Bild-/Tabellenbeschriftung neben dem Bild
\floatsetup[figure]{capbesideposition={inside,top}}
\floatsetup[table]{capbesideposition={inside,top},style=plaintop}
\renewfloatcommand{fcapside}{figure}[\capbeside][\FBwidth]
\newfloatcommand{tcapside}{table}[\capbeside][\FBwidth]


\selectlanguage{ngerman}


\deffootnote{1em}{1em}{%
 \makebox[1em][l]{\thefootnotemark}}

\makeindex

\newcommand{\real}{\mathord{\mathrm{I\!R}}}

\begin{document}
\selectlanguage{ngerman}
\def\figdir{figures}
\def\tabledir{tables}

\frontmatter

\pagestyle{scrplain}
\pagestyle{empty}

\begin{titlepage}

\sffamily

\raggedleft

\vspace*{-2cm}

\includegraphics{\figdir/logo-th-rosenheim-2019_master_quer_2c.eps}

\vfill

\centering
\LARGE
% \vspace*{\fill}
%-----------
Fakultät für Informatik  \vspace{0.5cm}\\
\Large
Studiengang Software- und Systems-Engineering

\vspace{2cm}

\LARGE

Erkennung von Design Patterns in Quellcode durch Machine Learning

\vspace{2cm}

\Large
Master Thesis

\vspace{1.5cm}


\Large
von

\vspace{0.5cm}

%\vspace*{\fill}

\LARGE
Mehmet Aslan \vspace{1cm}

\vspace{1cm}

\flushleft
 \Large
\vspace*{\fill}

%-----------
\begin{tabbing}
Datum der Abgabe: \= tt.mm.jjjj \kill
Datum der Abgabe: \> tt.mm.jjjj \\
Erstprüfer: \> Prof.\ Dr.\ Marcel Tilly\\
Zweitprüfer: \> Prof.\ Dr.\ Kai Höfig
\end{tabbing}
%-----------

\end{titlepage}

\cleardoubleemptypage

{
\large
\thispagestyle{empty}
\vspace*{\fill}

\noindent
\textsc{Eigenständigkeitserklärung / Declaration of Originality}

\medskip

\noindent
Hiermit bestätige ich, dass ich die vorliegende Arbeit selbständig verfasst und keine anderen als die angegebenen Hilfsmittel benutzt habe. Die Stellen der Arbeit, die dem Wortlaut oder dem Sinn nach anderen Werken (dazu zählen auch Internetquellen) entnommen sind, wurden unter Angabe der Quelle kenntlich gemacht.

\medskip

\textit{I declare that I have authored this thesis independently, that I have not used other than the declared sources / resources, and that I have explicitly marked all material which has been quoted either literally or by content from the used sources.}

\bigskip

\noindent
Rosenheim, den tt.mm.jjjj

\vspace*{2cm}

\noindent
Vor- und Zuname
}

%%% Local Variables: 
%%% mode: latex
%%% TeX-master: "d"
%%% End: 

\cleardoubleemptypage
\chapter*{Kurzfassung}
\thispagestyle{empty}

Design Patterns oder Entwurfsmuster sind in der Software-Entwicklung gängige Lösungsansätze für wiederkehrende Probleme.
Von Software-Entwicklern werden diese eingesetzt, um Probleme in der Implementierung oder in der Software-Architektur zu lösen, deren Lösungsweg bereits bekannt sind und für die jeweilige Situation abgeändert werden. Die wohl bekanntesten Design Patterns sind die 23 Entwurfsmuster nach Gamma et al.
Im weiteren Verlauf der Software-Entwicklung sind die eingesetzten Entwurfsmuster aufgrund iterativer Änderungen und steigender Komplexität im Quellcode nicht mehr einfach wiederzufinden.
Deswegen liegt der Fokus dieser Masterthesis auf der Etablierung eines Prozesses, welcher mit Machine Learning in der Lage ist, Design Pattern im Quellcode zu erkennen.

Das Ziel dieser Arbeit ist es, die Entwurfsmuster Singleton, Observer, Command und Adapter zu identifizieren. Im Kontext dieser Arbeit werden diese als Strukturen definiert, die in Summe von Rollen dargestellt werden.
Jedes dieser Rollen übernimmt im Kontext des jeweiligen Design Patterns unterschiedliche Aufgaben und Verantwortungen.
Dazu wird ein Klassifizierer trainiert, welches Rollen innerhalb dieser Entwurfsmuster durch Code-Metriken erkennt.
Anhand der Klassifikation der Rollen wird im finalen Schritt ein Übereinstimmungswert berechnet, der angibt, mit welcher Zuversicht es sich hierbei um das jeweilige Entwurfsmuster handelt.
Zum Schluss wird die Klassifikationsleistung der hier vorgestellten Methode anhand der Metriken \textit{Precision}, \textit{Recall} und \textit{f1} evaluiert. 

\bigskip

\noindent
Schlagworte: Software-Entwicklung, Machine Learning, Design Patterns 


\cleardoubleemptypage

\pagestyle{scrplain}
\pagenumbering{roman}
\include{toc}

\pagestyle{scrheadings}


\addtokomafont{caption}{\small}

\mainmatter

\chapter{Motivation}
\section{Einführung in Design Patterns}

\section{Untersuchungsfragen}

\chapter{Literaturrecherche}
\section{Design Patterns}
%%TODO

\subsection{Design Pattern Katalog}

\subsubsection{Creational Design Patterns}

\subsubsection{Structural Design Patterns}

\subsubsection{Behavioral Design Patterns}


\subsection{Rollenkatalog}
\include{chapters/research/other_approaches}
\section{Angewendete Ansätze mit Maschine Learning}
\section{Geeiegnetes Datensätze}

\subsection{Verfügbare gelabelte Datetsätze}

\subsection{Argumentieren des Datensatzes mit synthetischen Daten}
\section{Extrahierte Features}

Im Kontext dieser Arbeit werden Design Patterns als eine Summe von Substrukturen definiert, zu diesem jeweils eine Rolle zugeordnet wird.
Im Sinne dieser ist eine Menge an Erkennungsmerkmalen oder Features notwendig, welche Rückschlüsse auf die jeweilige Rolle ermöglicht. 
Dabei müssen sich die Aspekte für Struktur, Verhalten und Relation zu anderen Komponenten in ausgewählten Features widerspiegeln.
Die in einer einem vorherigen Abschnitt erläuterten Machine-Learning Verfahren nutzen verschiedene Arten von Features, um diese Aspekte zu encodieren.
Im Kern läuft darauf hinaus, die Erkennungsmerkmale in eine numerische Repräsentation zu transformieren, sodass ein Klasszifizier mit diesen umgehen kann.
Aus diesem Grund werden in dieser Arbeit die Merkmale der jeweiligen Instanzen der Rollen als numerische Vektoren dargestellt, die die Aspekte der Struktur, Verhalten und Relation als Metrriken encodieren.

Die Entscheidung, Metriken als Features für Rollen zu verwenden, bietet mehrere Vorteile. Zu einem ermöglicht der Einsatz von Metriken, Eigenschaften der Implemtierung der jeweiligen Rollen in einer abstrakten Form darzustellen.
Dadurch kann die Identifkation der Rollen unabhängig von Programmiersprache und Prokjektstruktur erfolgen. Zudem ist die Ermitellung von Metriken aufgrund möglichen Einsatz durch Software-Werkzeuge skalierbar. Die Ermittelung der jeweiligen Metriken erfolgt über statische Codeanlyse, welches durch Parallität in der Lage,
einen großen Umfang an Quelldateien in relativ kurzer Zeit zu bearbeiten. Der wohl größte Verteil, die Metriken erbringen, ist Transparenz und Reprodzierbarkeit.
Bevor ein Wert zu einer Metrik ermitellt werden kann, muss vornehinein definiert werden, was die jeweilige Metrik aussagt und wie diese ermitellt wird. Durch eine automatisierte und strikte Befolgung der Definition der Berechnung kann durch statische Codeanalyse in mehreren Durchläufen der gleiche Wert für die Metrik im Kontext der Rolleninstanz bestimmt werden.
Dadruch ensteht ebenfalls eine Transparenz, wie die Werte ermitellt werden. Bei den hier erläutern Verfahren, die die Features der Entiäten aus dem Quellcode graphisch oder in natrlicher Sprache enkodiert haben, müssen diese zuätzlich in eine numerische Form gebracht werden.
Die erfolgt meist durch die Inkludierung eines zusätzlichen Modells. Zu einem erfordert dies zusätzliches Training und Validierung des Modells, welches je nach Modell und Umfang das Datensatzes auch noch Zeit und Rechnenleistung in Anspruch nimmt. Dazu ist die Leistung des Klassfizieres abhängig von der Qualtiät der Resulate des enkodierenden Modells. Mögliche Leistungsdefizite des Enkodierens in numerische Werte propagiert zu der Klassifizierungskapazität des gesamten Verfahrens.
Zwar ist die Ermittlung von Metriken davon nicht ausgeschlossen, jedoch sind Rückschlüsse möglich wie diese Werte zustande gekommen sind. Dahingegen sind die Modelle, die die Features transformieren, als Black Box zu betrachten. Die Werte innerhalb des Modells, die die Transformation der Eingabedaten beeinflüssen, sind für das Menschenauge meist nur schwer verstehbar und bieten keinen Rückschluss, wie diese berechnet worden.
Zusammenfassnd ermöglichen Metriken als Feature eine Unabhängigkeit von Programmiersprache und Projektstruktur, Skalierbarkeit, Reproduzierbarkeit und Transparenz, die bei dem weiteren Verfeinern der hier vorgestellten Methodik wünschenswert sind.

\pagebreak

Um die Kennmerkmale hier zu ermitelln, erfordert dies ein Katalog an Metriken, die angegeben, wie diese ermitellt werden und was diese aussagen. Hierbei werden Metriken verwendet, die die Aspekte der Struktur, Verhalten und Relation widerspiegeln.
Dazu werden Metriken verwendet, die bereists in anderen Arbeiten verwendet wurden, die die Struktur repräsentieren und andere Metriken von Chidamber et al., die Einblicke in Verhalten und Relation bieten~\cite{chidamber1994metrics}:


\begin{table}[H]
    \begin{tabular}{|c|p{0,45\linewidth}|p{0,35\linewidth}|c|}
        \hline
        Kürzel & Langform &Beschreibung & Datentyp\\
        \hline
        COA & COUNT\_OF\_ABSTRACT\_METHODS & Anzahl de abstrakten Methoden in einer Klasse & int\\
        COF & COUNT\_OF\_FIELDS & Anzahl der Felder in einer Klasse & int\\
        COM & COUNT\_OF\_METHODS & Anzahl der Methoden in einer Klasse & int\\
        COF & COUNT\_OF\_OBJECT\_FIELDS &Anzahl der Felder in einer Klasse mit Objekttyp & int\\
        COPC & COUNT\_OF\_PRIVATE\_CONSTRUCTORS & Anzahl der privaten Konstruktoren & int\\
        COPF & COUNT\_OF\_PRIVATE\_FIELDS & Anzahl der privaten Felder & int\\
        COSF & COUNT\_OF\_STATIC\_FIELDS & Anzahl an statischen Feldern & int\\
        CBO & COUPLING\_BETWEEN\_OBJECTS & Anzahl der Abhänigkeiten in der Klasse nach Chidamer et al. & int\\
        DIT & DEPTH\_OF\_INHERITANCE & Anzahl der Elterklassen einer Klasse nach Chidamer et al. & int\\
        RPC & RESPONSE\_FOR\_A\_CLASS & Anzahl der Methoden, die direkt durch Klassenmethoden aufgerufen werden nach Chidamer et al. & int\\
        WMC & WEIGHTED\_METHOD\_CLASS & Zyklmatische Komplexität nach McCabe~\cite{mccabe1976complexity} & float\\
        \hline
    \end{tabular}
    \caption{Ausgewählte Metriken als Features}
\end{table}


\section{Betrachte Multiclass-Klasszifizierer}

\subsection{Ein-Modell Architekturen}

\subsection{Mehr-Modell Architekturen}
\section{Metriken für Klasszifizierer}

\subsection{F1}

\subsection{Recall}

\subsection{Precision}

\subsection{ROC}
\section{Angewendete Technologien, Frameworks und Bibliotheken}

\chapter{Methodologie}
\section{Geeiegnetes Datensätze}

\subsection{Verfügbare gelabelte Datetsätze}

\subsection{Argumentieren des Datensatzes mit synthetischen Daten}
\section{Extrahierte Features}

Im Kontext dieser Arbeit werden Design Patterns als eine Summe von Substrukturen definiert, zu diesem jeweils eine Rolle zugeordnet wird.
Im Sinne dieser ist eine Menge an Erkennungsmerkmalen oder Features notwendig, welche Rückschlüsse auf die jeweilige Rolle ermöglicht. 
Dabei müssen sich die Aspekte für Struktur, Verhalten und Relation zu anderen Komponenten in ausgewählten Features widerspiegeln.
Die in einer einem vorherigen Abschnitt erläuterten Machine-Learning Verfahren nutzen verschiedene Arten von Features, um diese Aspekte zu encodieren.
Im Kern läuft darauf hinaus, die Erkennungsmerkmale in eine numerische Repräsentation zu transformieren, sodass ein Klasszifizier mit diesen umgehen kann.
Aus diesem Grund werden in dieser Arbeit die Merkmale der jeweiligen Instanzen der Rollen als numerische Vektoren dargestellt, die die Aspekte der Struktur, Verhalten und Relation als Metrriken encodieren.

Die Entscheidung, Metriken als Features für Rollen zu verwenden, bietet mehrere Vorteile. Zu einem ermöglicht der Einsatz von Metriken, Eigenschaften der Implemtierung der jeweiligen Rollen in einer abstrakten Form darzustellen.
Dadurch kann die Identifkation der Rollen unabhängig von Programmiersprache und Prokjektstruktur erfolgen. Zudem ist die Ermitellung von Metriken aufgrund möglichen Einsatz durch Software-Werkzeuge skalierbar. Die Ermittelung der jeweiligen Metriken erfolgt über statische Codeanlyse, welches durch Parallität in der Lage,
einen großen Umfang an Quelldateien in relativ kurzer Zeit zu bearbeiten. Der wohl größte Verteil, die Metriken erbringen, ist Transparenz und Reprodzierbarkeit.
Bevor ein Wert zu einer Metrik ermitellt werden kann, muss vornehinein definiert werden, was die jeweilige Metrik aussagt und wie diese ermitellt wird. Durch eine automatisierte und strikte Befolgung der Definition der Berechnung kann durch statische Codeanalyse in mehreren Durchläufen der gleiche Wert für die Metrik im Kontext der Rolleninstanz bestimmt werden.
Dadruch ensteht ebenfalls eine Transparenz, wie die Werte ermitellt werden. Bei den hier erläutern Verfahren, die die Features der Entiäten aus dem Quellcode graphisch oder in natrlicher Sprache enkodiert haben, müssen diese zuätzlich in eine numerische Form gebracht werden.
Die erfolgt meist durch die Inkludierung eines zusätzlichen Modells. Zu einem erfordert dies zusätzliches Training und Validierung des Modells, welches je nach Modell und Umfang das Datensatzes auch noch Zeit und Rechnenleistung in Anspruch nimmt. Dazu ist die Leistung des Klassfizieres abhängig von der Qualtiät der Resulate des enkodierenden Modells. Mögliche Leistungsdefizite des Enkodierens in numerische Werte propagiert zu der Klassifizierungskapazität des gesamten Verfahrens.
Zwar ist die Ermittlung von Metriken davon nicht ausgeschlossen, jedoch sind Rückschlüsse möglich wie diese Werte zustande gekommen sind. Dahingegen sind die Modelle, die die Features transformieren, als Black Box zu betrachten. Die Werte innerhalb des Modells, die die Transformation der Eingabedaten beeinflüssen, sind für das Menschenauge meist nur schwer verstehbar und bieten keinen Rückschluss, wie diese berechnet worden.
Zusammenfassnd ermöglichen Metriken als Feature eine Unabhängigkeit von Programmiersprache und Projektstruktur, Skalierbarkeit, Reproduzierbarkeit und Transparenz, die bei dem weiteren Verfeinern der hier vorgestellten Methodik wünschenswert sind.

\pagebreak

Um die Kennmerkmale hier zu ermitelln, erfordert dies ein Katalog an Metriken, die angegeben, wie diese ermitellt werden und was diese aussagen. Hierbei werden Metriken verwendet, die die Aspekte der Struktur, Verhalten und Relation widerspiegeln.
Dazu werden Metriken verwendet, die bereists in anderen Arbeiten verwendet wurden, die die Struktur repräsentieren und andere Metriken von Chidamber et al., die Einblicke in Verhalten und Relation bieten~\cite{chidamber1994metrics}:


\begin{table}[H]
    \begin{tabular}{|c|p{0,45\linewidth}|p{0,35\linewidth}|c|}
        \hline
        Kürzel & Langform &Beschreibung & Datentyp\\
        \hline
        COA & COUNT\_OF\_ABSTRACT\_METHODS & Anzahl de abstrakten Methoden in einer Klasse & int\\
        COF & COUNT\_OF\_FIELDS & Anzahl der Felder in einer Klasse & int\\
        COM & COUNT\_OF\_METHODS & Anzahl der Methoden in einer Klasse & int\\
        COF & COUNT\_OF\_OBJECT\_FIELDS &Anzahl der Felder in einer Klasse mit Objekttyp & int\\
        COPC & COUNT\_OF\_PRIVATE\_CONSTRUCTORS & Anzahl der privaten Konstruktoren & int\\
        COPF & COUNT\_OF\_PRIVATE\_FIELDS & Anzahl der privaten Felder & int\\
        COSF & COUNT\_OF\_STATIC\_FIELDS & Anzahl an statischen Feldern & int\\
        CBO & COUPLING\_BETWEEN\_OBJECTS & Anzahl der Abhänigkeiten in der Klasse nach Chidamer et al. & int\\
        DIT & DEPTH\_OF\_INHERITANCE & Anzahl der Elterklassen einer Klasse nach Chidamer et al. & int\\
        RPC & RESPONSE\_FOR\_A\_CLASS & Anzahl der Methoden, die direkt durch Klassenmethoden aufgerufen werden nach Chidamer et al. & int\\
        WMC & WEIGHTED\_METHOD\_CLASS & Zyklmatische Komplexität nach McCabe~\cite{mccabe1976complexity} & float\\
        \hline
    \end{tabular}
    \caption{Ausgewählte Metriken als Features}
\end{table}


\include{chapters/methodolgy/applied_classifiered}
\section{Evaluation des trainierten Models}
\section{Modellauswahl, Training und Validierung}
Nach dem bestimmen eines passenden Datensatzes und zu extrahierenden Features wird ein Klassifizierer trainiert, welcher mit einem Wahrschenlichkeitswert angibt, welche Rolle am ehesten der Quellcodeentiät zugrodnet werden kann.
Dazu wird in dieser Sektion diskutiert, welche Modelle betrachtet werden und wie diese traniert und evaluiert werden können. 


\subsection*{Auswahl der Klassifizierer}
Im Zuge dieser Arbeit werden in der Sektion \ref{classifiers} erläuterten Klassifizierer als die bevorzugten Modelle hergenommen. Diese Auswahl der Modelle ist unterschiedliche Gründe zurückzuführen.
Klassifizierer können je nach Art und Weise, wie sie innerlich funktionieren, in Kategorien untergliedert werden. Dabei dienen diese Modelle als Repräsentant, sodass ein Spektrum von unterschiedlich funktionierenden Modellen getestet werden kann.
Ohne das Testen des jeweiligen Models ist dessen Leistung für das in dieser Arbeit genutzten Datensatz nicht vorherzusagen. Zudem können Implementierungen der jeweiligen Modelle in verschiedenen Programmiersprachen für verschiedene Plattformen vorgefunden werden.
Dies ermöglicht die Nutzung in Programmiersprachen wie Python oder R, welche für Maschine Learning bevorzugt werden. Das Software-Framework, welches die Implementierungen für die Modelle zur Verfügung stellt und im Kontext dieser Arbeit verwendet wird, wird im weiteren Verlauf der Arbeit genauer erläutert.
Der größte Vorteil dieser Klassifizierer ist deren Simplizität und damit resultierende Effizienz in Zeit- und Speicherkomplexität. Da die Klassifizierer im weiteren Verlauf der Methode durch den Einsatz von Hyperparameter-Tuning iterativ optimiert werden, ist dies von besonderer Bedeutung.
Deshalb werden Modelle mit komplexerer Architektur aufgrund des Mangels verfügbarer Rechenressourcen zu dem Verfassen der Arbeit nicht weiter betrachtet. 
Die Auswahl der Klasszfizier aus Sektion~\ref{classifiers} beantwortet die Unterschungsfrage~\ref{RQ5}


\subsection*{Training der Modelle}
In dieser Arbeit wird das Trainieren der Modelle mit dem Hyperparameter-Tuning aus Sektion~\ref{hyper_params} gekoppelt. Zuerst wird der Datensatz in ein Trainings- und Validationsdatensatz aufgeteilt. Danach wird für jedes Modell ein Suchraum für die Hyperparameter-Werte, Anzahl der Trainingsiterationen und die Metrik bestimmt, wonach die Leistung des Klassifizierers
bestimmt wird. In jeder Iteration wird zufällig eines der hier vorgestellten Klassifizierer mit einer vorher bestimmten Hyperparameter-Konfiguration instanziiert, mit dem Trainingsdatensatz trainiert und durch Einsatz des Validationsdatensatz der Wert der Leistungsmetrik bestimmt. Die Hyperparameter-Konfiguration der Instanz mit der besten Leistung aus allen Iterationen wird im weiteren Verlauf der Arbeit verwendet.

\subsection*{Validierung des Modells}
Dadruch, dass der Datensatz für das Training aufgeteilt wird, wird die beste Hyperparameter-Konfiguration eines der hier vorgestellten Klassifizierer durch Kreuzvalidierung zusätzlich validiert. Dies dient zu Evaluation des Modells in der jetzigen Konfigiration.
Dabei wird der gesamte verfügbare Datensatz verwendet. Die Evaluierung der Leistung des Klassfikationskapazität erfolgt mit der gleichen Metrik wie in der Trainingsphase und den restlichen in Sektion~\ref{metrics}. Es bestünde die Option, dass man die Kreuzvalidierung direkt in die Trainingsphase integriert. 
Jedoch ist dabei zu beachten, dass die Kreuzvalidierung selbst iterativ agiert und dies in jeder Iteration des Hyperparameter-Tunings durchgeführt wird. Dies erhöht die Laufzeit der Trainingsphase signifikant.


\include{chapters/methodolgy/evalution_methodology}

\chapter{Zukünftige Aussichten}
\section{Zukünftige Aussichten}

Die fortschrittlichen Entwicklungen im Bereich des maschinellen Lernens (ML) und speziell der Large Language Models (LLMs) eröffnen neue Perspektiven für die automatisierte Erkennung von Design Patterns in Quellcode. In dieses Feld wurden zum Zeitpunkt der Verfassen der Arbeit Fortschritte getätigt, da LLMs das Potenzial geben, die Effizienz und Genauigkeit bei der Identifizierung von Design Patterns erheblich zu verbessern. Im Folgenden werden die zukünftigen Aussichten dieser Technologie beleuchtet.

\begin{enumerate}
    \item \textbf{Erweiterte Erkennungskapazitäten}: Mit der zunehmenden Verfeinerung von LLMs ist zu erwarten, dass ihre Fähigkeit, komplexe Muster und Abstraktionen im Quellcode zu erkennen, deutlich zunimmt. Diese Modelle können aus einer umfangreichen Datenmenge lernen und somit eine breite Palette von Design Patterns identifizieren, die in verschiedenen Programmiersprachen und -stilen zum Einsatz kommen. Die Flexibilität von LLMs ermöglicht es ihnen, auch seltene oder weniger dokumentierte Patterns zu erkennen, die herkömmliche Methoden möglicherweise übersehen.
    \item \textbf{Verbesserung der Präzision und Reduzierung von Fehlalarmen}: Durch das Training mit großen Datensätzen können LLMs nicht nur eine Vielzahl von Patterns erkennen, sondern auch den Kontext, in dem diese Patterns verwendet werden, besser verstehen. Dies führt zu einer höheren Präzision bei der Erkennung und einer signifikanten Reduzierung von Fehlalarmen. Die Fähigkeit, den Kontext zu berücksichtigen, ist besonders wichtig, da viele Design Patterns nur in bestimmten Situationen angemessen sind. LLMs können feine Unterschiede im Code erfassen, die darauf hinweisen, ob ein bestimmtes Pattern tatsächlich beabsichtigt ist oder nicht.
    \item \textbf{Automatisierte Verbesserungsvorschläge und Refactoring}: Zukünftige Entwicklungen könnten LLMs befähigen, nicht nur existierende Patterns zu erkennen, sondern auch Verbesserungsvorschläge zu machen. Basierend auf der erkannten Implementierung eines Design Patterns könnten diese Modelle Empfehlungen für ein effizienteres oder klareres Pattern geben, das in den aktuellen Code eingeführt werden kann. Darüber hinaus ist das Potenzial für automatisiertes Refactoring beträchtlich, wobei LLMs Vorschläge für Code-Umstrukturierungen machen können, um Design Principles wie SOLID besser einzuhalten.
    \item \textbf{Interaktive Entwicklungsumgebungen}: Die Integration von LLMs in Entwicklungsumgebungen und IDEs (Integrated Development Environments) könnte zu einer interaktiveren und unterstützenden Codierungserfahrung führen. Entwickler könnten in Echtzeit Feedback zu den von ihnen verwendeten Design Patterns erhalten, einschließlich Hinweisen zur Anwendung und möglichen Optimierungen. Diese Art der direkten Integration fördert ein tieferes Verständnis für Design Patterns und unterstützt Entwickler dabei, best practices effektiver in ihren Code zu integrieren.
\end{enumerate}

\chapter*{Einführung}

\section*{Einführung in Design Patterns}

Bei der Entwicklung von Software-System stößt man auf Probeleme und Situationen, denen man bereits in der gleichen Form oder in einer ähnlichen Variation entgegenkommen ist.
Daher tendiert man, einen bereits bekannten Lösungsansatz hier wieder einzusetzen. Die Idee der Wiederverwendbarkeit von bekannten Lösungsanätzen für wiederaufkehrende zu lösende Probleme in der Software-Entwicklung
von Software-Systemen stellen den Kern der Entwurfsmunster oder \textit{Design Patterns} dar. Allgemeiner betrachtet bilden Entwurfsmunster eine Art Lösungsblaupause für wiederaufkehrende Probleme dar, die für die jeweilige Situation angepasst werden.
 \textit{Design Pattern - Elements of Reusable Object-Oriented Software} von Gamma et. al \cite{gamma1994design} ist wohl das bekannteste Werk, das sich mit dieser Thematik auseinander setzt.
Normalerweise werden Design Patterns verwendet, um abstrakte Probleme in der Software-Architektur von objekt-orientierten Programmiersprachen zu lösen, die sich mit der Kreation, Struktur oder Verhalten auseinandersetzen.
Obwohl Design Patterns als bekannte und erpropte Blaupause für wiederaufkehrende Probleme in Software-Systemen eingesetzt werden können, bringen Design Patterns Konsequenzen für die zu betrachtende Situation mit sich,
die in Erwägung gezogen werden sollten. Diese sind meistens erhöhte Zeit- und Speicherkomplexität als Austausch für eine bessere Flexibilität im Design des Software-Systems, so dass dessen Adaptierbarkeit für neue oder sich änderene Anforderungen sichergestellt werden können.

Die Beschreibung von Design Patterns sind detaliert und beinhalten unteranderem dessen Namen, Absicht, Motivation, Awnwendbarkeit, Struktur, Teilnehmer und deren Zusammenarbeit und andere Informationen. Die Semantik des Design Patterns inkludieren die Abischt, Motivation und Anwendbarkeit,
die angeben, was das Entwurfsmuster macht, warum dieses benötigt wird und wo es sinnvoll engegesetzt werden kann. Die Teilnehmer reflektieren deren Natur als Blaupause, da diese Rollen darstellen, die die Klassen im Kontext des Design Patterns darstellen, während die Struktur und die Zusammenarbeit zwischen den Teilnehmern die Interaktionen zwischen diesen beschreibt.
Diese Menge an Informationen ist in den Entwurfsmustern un deren Beschreibung enkodiert und wird durch Software-Entwickler in das Software-System während dessen Implementierung engeflochtet. Der Einsatz von Design Patterns ist auf Design-Entscheidung während der Entwicklung des Software-Systems zurückzuführen und des Öfteren wird diese und die jewiligen Hintergrundgedanken nicht weiter dokumentiert.
Davon ebenfalls betroffen sind Anpassungen und konkrete Implementierungsdetails des Entwurfsmusters. Schlussendlich tauchen diese Einzelheiten in Quellcode des Software-Systems ab. Das Widerfinden dieser enkodierten Information im Quellcode für die Zwecke der Weiterentwicklung und Wartung ist der Hauptmotivator für die Erkennung von Entwurfsmusterm.



\appendix
\chapter{Zusätzliche Tabellen für die Datenanalyse von P-MArt}


\begin{table}[H]
    \centering
    \begin{tabular}{|c|c|}
        \hline
        Rolle & Aufkommen in Datensatz\\
        \hline
        abstractclass & 7\\abstractfactory & 2\\abstraction & 1\\abstractproduct & 3\\adaptee & 17\\adapter & 29\\aggregate & 5\\builder & 1\\caretaker & 5\\client & 64\\colleague & 2\\command & 6\\component & 7\\composite & 11\\concreatecolleague & 4\\concreteaggregate & 6\\concretebuilder & 14\\concreteclass & 51\\concretecommand & 50\\concretecomponent & 55\\concretecreator & 13\\concretedecorator & 6\\concreteelement & 102\\concretefactory & 8\\concreteimplementor & 4\\concreteiterator & 8\\concretemediator & 2\\concreteobserver & 28\\concreteproduct & 38\\concreteprototype & 3\\concretestate & 21\\concretestrategy & 68\\concretesubject & 59\\concretevisitor & 29\\context & 24\\creator & 6\\decorator & 1\\director & 4\\element & 5\\facade & 1\\implementor & 1\\invoker & 29\\iterator & 4\\leaf & 92\\mediator & 2\\memento & 8\\nullobject & 1\\objectstructure & 4\\observer & 9\\originator & 8\\product & 46\\prototype & 1\\proxy & 5\\realsubject & 5\\receiver & 12\\refinedabstraction & 3\\singleton & 15\\state & 5\\strategy & 6\\subject & 9\\subsystemclass & 10\\target & 10\\visitor & 5\\
        \hline
    \end{tabular}
    \caption{tabellarische Darstellung der Rollenverteilung in P-MArt}
\end{table}

%%% Local Variables: 
%%% mode: latex
%%% TeX-master: "thesis.tex"
%%% End: 


\cleardoublepage

\bibliographystyle{natger}
\bibliography{thesis}

\cleardoublepage


\footnotesize
\printindex


\end{document}
