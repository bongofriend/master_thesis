\chapter{Zusätzliche Tabellen für die Datenanalyse von P-MArt}


\begin{table}[H]
    \centering
    \begin{tabular}{|c|c|}
        \hline
        Rolle & Aufkommen in Datensatz\\
        \hline
        abstractclass & 7\\abstractfactory & 2\\abstraction & 1\\abstractproduct & 3\\adaptee & 17\\adapter & 29\\aggregate & 5\\builder & 1\\caretaker & 5\\client & 64\\colleague & 2\\command & 6\\component & 7\\composite & 11\\concreatecolleague & 4\\concreteaggregate & 6\\concretebuilder & 14\\concreteclass & 51\\concretecommand & 50\\concretecomponent & 55\\concretecreator & 13\\concretedecorator & 6\\concreteelement & 102\\concretefactory & 8\\concreteimplementor & 4\\concreteiterator & 8\\concretemediator & 2\\concreteobserver & 28\\concreteproduct & 38\\concreteprototype & 3\\concretestate & 21\\concretestrategy & 68\\concretesubject & 59\\concretevisitor & 29\\context & 24\\creator & 6\\decorator & 1\\director & 4\\element & 5\\facade & 1\\implementor & 1\\invoker & 29\\iterator & 4\\leaf & 92\\mediator & 2\\memento & 8\\nullobject & 1\\objectstructure & 4\\observer & 9\\originator & 8\\product & 46\\prototype & 1\\proxy & 5\\realsubject & 5\\receiver & 12\\refinedabstraction & 3\\singleton & 15\\state & 5\\strategy & 6\\subject & 9\\subsystemclass & 10\\target & 10\\visitor & 5\\
        \hline
    \end{tabular}
    \caption{tabellarische Darstellung der Rollenverteilung in P-MArt}
\end{table}

%%% Local Variables: 
%%% mode: latex
%%% TeX-master: "thesis.tex"
%%% End: 
